% Appendix A

\chapter{Software Documentation} % Main appendix title

\label{Appendix-PreUnity} % For referencing this appendix elsewhere, use \ref{AppendixA}
\begin{itemize}
\item \textbf{FDS2PreUnityTool}
\item[]
Converts the plot-3D output from Fire Dynamic Simulator (FDS) in to a more optimized data format to be visualized inside of unit. 
   \begin{itemize}
        \item \textbf{fds2ComplexGeom.py}
        \item[] Converts fds input file into a JSON with tree location data and complex terrain information
        \item \textbf{fdsOutput2Unity.py}
        \item[] Converts plot3D data from fds into binary data to be loaded into unity
                    \item[] \textbf{binary File Structure}   
                \begin{itemize}
                \item\textbf{Header}
                \item[] Five 32-bit integers corresponding to the length of each of the five scalar values from the plot-3d output
                \item\textbf{ 2D array with [x,y,z,q] as quantities, where x, y, and z represent the position while q represents the respective scalar value } 
                \item[Note:] this is a space matrix version of the output from plot-3D files
                \end{itemize}
        \item \textbf{fdsPathLines.py}
        \item[] Takes plot3D data and calculate the amount of turbulence (represented as Reynolds numbers) to identify regions of interest. Then calculates pathlines that will intersect with points sampled from these regions
        \item \textbf{main.py}
        \item[] Creates the p[roper file structure and ensures all functions save data into the proper location  
   \end{itemize}
\pagebreak

\item \textbf{FDS2UnityTool}
   \begin{itemize}
        \item \textbf{ConfigData.cs}
        \item[] Saves all information when transitioning from one scene to another 

        \item \textbf{hrrLoader.cs}
        \item[] Loads in a color mapped and animated fire particle engine in each voxel per timestep
        \item \textbf{MainMenu.cs}
        \item[] Verifies menu check boxes are selected and verifies fire and directory structure for input data
        \item \textbf{PauseMenu.cs}
        \item[] Checks pause button input and tracks time in pause menu to allow for accurate timesteps after pausing 
        \item \textbf{smokeLoader.cs}
        \item[] Loads in a color mapped and animated smoke particle engine in each voxel per timestep
        \item \textbf{smvReader.cs}
        \item[] Loads in and cache's heat release rate and smoke density data additionally tracking current timestep
        
        \item \textbf{tduController.cs}
        \item[] Uses calulations of thermal dosage from Parsons, Russell A., Butler, Bret. W., and Mell, William “Ruddy" to visualize burn injury potential \cite{Parsons2014}
        \item \textbf{TerrainBuilder.cs}
        \item[] Loads JSON of terrain data from fds2ComplexGeom.py \ref{PythonComplexGeom} and maps to a 3D mesh with texturing
        \item \textbf{TreeLoader.cs}
        \item[] Loads JSON of tree data from  fds2ComplexGeom.py \ref{PythonComplexGeom} and maintains scale regardless of the size of the 3D tree object        
        \item \textbf{WindStreams.cs}
        \item[] Loads in and cache's Pathline data from fdsPathLines.py \ref{PythonPathLines} to be visualized each timestep
        \end{itemize}
\end{itemize}
\pagebreak

\section{FDS2UnityVR}
\subsection{fds2ComplexGeom.py}
Converts fds input file into a JSON with tree location data and complex terrain information
\label{PythonComplexGeom}
\begin{itemize}
    \item \textbf{Class instantiation}
    \begin{itemize}    
    \item Parameters   
        \begin{itemize}
        \item\textbf{fds\_input\_location}: complete path to fds input file (.fds)
        \item \textbf{tree\_id}: name of the tree label
        \item \textbf{non\_terrain\_obsts}: list of any obstacles that are not classified as terrain.
        \end{itemize} 
    \end{itemize}
    
    \item \textbf{read\_in\_fds\_obst}
    \item[] Reads in fds input file and generates a dictionary of dictionary's containing the elevation data for the terrain
    
    \item \textbf{read\_in\_tree\_locations}
    \item[] Reads in all the locations of trees in the fds input file and parses other tree information
      
    \item \textbf{read\_in\_fds\_mesh}
    \item[]  Reads in fds input file and parses the MESH information
    
    \item \textbf{complex\_geom}
    \item[] Takes read in topography information and formats it for a 3D mesh and builds a 3-D mesh

    \begin{itemize}    
    \item Variables   
        \begin{itemize}
        \item\textbf{vertices}: Nx3 numpy array of all vertex points
        \item \textbf{vert\_counter}: keeps track of current vertex index
        \item \textbf{faces}: array of all face indexes
        \item \textbf{face\_counter}: keeps track of current face index (one indexed)
        \item \textbf{col\_keys}: reverse sorted list of all index points of columns
        \item \textbf{row\_keys}: reverse sorted list of all index points of current row
        
        \end{itemize} 
    \end{itemize}

    \item \textbf{save\_to\_json}
    \item[] Saves all parsed data into a JSON file
                \item[] \textbf{JSON File Structure}   
                \begin{itemize}
                \item\textbf{meshData}
                    \begin{itemize}
                    \item\textbf{X\_MIN:} x value of the origin point of the mesh 
                    \item\textbf{Y\_MIN:} y value of the origin point of the mesh 
                    \item\textbf{Z\_MIN:} z value of the origin point of the mesh 
                    \item\textbf{X\_MAX:} x value of the opposite corner
                    \item\textbf{Y\_MAX:} y value of the opposite corner
                    \item\textbf{Z\_MAX:} z value of the opposite corner
                    \item\textbf{I:} number of cells in the x direction
                    \item\textbf{J:} number of cells in the y direction
                    \item\textbf{K:} number of cells in the z direction
                    \end{itemize}
                \item\textbf{verts} 
                \item[] List of all vertices of the topographical map
                \item\textbf{faces} 
                \item[] List of the indexed of vertices to form triangle faces (One indexed values)
                \item\textbf{treeList}
                    \begin{itemize}
                    \item\textbf{x:} x position
                    \item\textbf{y:} y position
                    \item\textbf{height:} ground height at position x,y
                    \item\textbf{crownBaseHeight:} Height from ground level where tree crown begins
                    \item\textbf{crownRadius:} Radius of tree crown
                    \item\textbf{crownHeight:} Height of tree crown 
                    \end{itemize}
                \end{itemize} 
    \item \textbf{write\_hdf5}
    \item[] Saves all parsed data into a HDF5 file 
    \item[Note:] Reading in the complex geometry as a HDF5 file has not been implemented in to the current Unity project
                \item[] \textbf{hdf5 File Structure}   
                \begin{itemize}
                \item\textbf{meshData} Array of nine, 32-bit floats
                \item[] [X\_MIN,Y\_MIN, Z\_MIN, X\_MAX, Y\_MAX, Z\_MAX, I, J, K]
                    \begin{itemize}
                    \item\textbf{X\_MIN:} x value of the origin point of the mesh 
                    \item\textbf{Y\_MIN:} y value of the origin point of the mesh 
                    \item\textbf{Z\_MIN:} z value of the origin point of the mesh 
                    \item\textbf{X\_MAX:} x value of the opposite corner
                    \item\textbf{Y\_MAX:} y value of the opposite corner
                    \item\textbf{Z\_MAX:} z value of the opposite corner
                    \item\textbf{I:} number of cells in the x direction
                    \item\textbf{J:} number of cells in the y direction
                    \item\textbf{K:} number of cells in the z direction
                    \end{itemize}
                \item\textbf{verts} 
                \item[] List of all vertices of the topographical map
                \item\textbf{faces} 
                \item[] List of the indexed of vertices to form triangle faces (One indexed values)
                \item\textbf{treeList} 2D array Size N(umber of trees) by 6
                \item[] [x,y,height,crownBaseHeight, crownRadius, crownHeight]
                    \begin{itemize}
                    \item\textbf{x:} x position
                    \item\textbf{y:} y position
                    \item\textbf{height:} ground height at position x,y
                    \item\textbf{crownBaseHeight:} Height from ground level where tree crown begins
                    \item\textbf{crownRadius:} Radius of tree crown
                    \item\textbf{crownHeight:} Height of tree crown 
                    \end{itemize}
                \end{itemize} 
\end{itemize}

\subsection{fdsOutput2Unity.py}
\label{PythonOutput2Unity}
Converts plot3D data from fds into hdf5 data to be loaded into unity
\begin{itemize}
    \item \textbf{Class instantiation}
    \begin{itemize}    
    \item Parameters   
        \begin{itemize}
        \item\textbf{fds\_output\_directory}:  Full path to where fds data was output
        \item\textbf{fds\_input\_location}:  Full path to fds input file
        \item\textbf{save\_location}: Full path to where data should be saved once processed
        \item\textbf{save\_type}: Allows for selection for output filetype
.
        \end{itemize} 
    \end{itemize}
    
    \item \textbf{read\_in\_fds}
    \item[] Reads in fds input file to generate list of data being saved as plot3D 
    
    \item \textbf{get\_file\_timestep}
    \item[] Extracts the timestep from the filename
      
    \item \textbf{group\_files\_by\_time}
    \item[]  Groups filenames by timestamp to allow for multi mesh simulations
    
    \item \textbf{find\_max\_values\_parallel}
    \item[] Parallel function to find the max values for each time of plot3D Data

    \item \textbf{get\_values}
    \item[] Calculates minimum and max values for all types of plot3D Data


    \item \textbf{runParallel}
    \item[] Reads in and saves all plot3D data in parallel.

    \item \textbf{get\_mesh\_number}
    \item[] Pulls mesh number from filename

    \item \textbf{q\_file\_to\_dict}
    \item[] Reads in plot3D data and saves it to a dictionary to allow for all threads to access it when running in parallel 

    \item \textbf{write\_to\_json}
    \item[] Saves all data to JSON format
        \item[] \textbf{JSON File Structure}   
                \begin{itemize}
                \item\textbf{fire:} 2D array, size: N {number of voxels}  by 4
                \item[] [x, y, z, q] where x, y, and z represent the position while q represents the heat release rate
                \item\textbf{smoke:} 2D array, size: N {number of voxels}  by 4
                \item[] [x, y, z, q] where x, y, and z represent the position while q represents the smoke density
                \item\textbf{configData:}
                \begin{itemize}
                    \item\textbf{min:} 
                \end{itemize} 
    
                \end{itemize} 
    

                "fire": current_fire_array,
                "smoke": current_density_array,
                "configData": min_max_dict,

    \item \textbf{write2bin}
    \item[] Saves all data in a custom binary format
    \item \textbf{write\_to\_hdf5}
    \item[] Saves all data in HDF5 format


\end{itemize}


\subsection{fdsPathLines.py}
\label{PythonPathLines}
Takes plot3D data and calculate the amount of turbulence (represented as Reynolds numbers) to identify regions of interest. Then calculates pathlines that will intersect with points sampled from these regions
\begin{itemize}
    \item \textbf{Class instantiation}
    \begin{itemize}    
    \item Parameters   
        \begin{itemize}
        \item\textbf{directory}: location of the fds output files
        \item\textbf{fds\_input\_location}: location of the fds input file.
        \end{itemize} 
 
    \end{itemize}   
    
    \item \textbf{\_\_check\_valid\_file}
    \item[] Calculates and saves voxel size
    
    \item \textbf{\_\_set\_voxel\_size}
    \item[] Extracts the timestep from the filename
      
    \item \textbf{group\_files\_by\_time}
    \item[]  Groups filenames by timestamp to allow for multi mesh simulations
    
    \item \textbf{find\_max\_values\_parallel}
    \item[] Parallel function to find the max values for each time of plot3D Data

    \item \textbf{get\_values}
    \item[] Calculates minimum and max values for all types of plot3D Data


    \item \textbf{runParallel}
    \item[] Reads in and saves all plot3D data in parallel.

    \item \textbf{get\_mesh\_number}
    \item[] Pulls mesh number from filename

    \item \textbf{q\_file\_to\_dict}
    \item[] Reads in plot3D data and saves it to a dictionary to allow for all threads to access it when running in parallel 

    \item \textbf{write\_to\_json}
    \item[] Saves all data to JSON format

    \item \textbf{write2bin}
    \item[] Saves all data in a custom binary format
    \item[] \textbf{hdf5 File Structure}   
                \begin{itemize}
                \item\textbf{maxValue:} Maximum magnitude of wind represented as a pathline in the current timestep
                \item\textbf{number\_of\_wind\_streams:} Number of pathlines visualized in the current timestep
                \item\textbf{length\_of\_wind\_streams:} Array the size of the number of wind streams indicating the number of line segments per pathline
                \item\textbf{windstream\_\{N\}:} 2D array, size: N {length of current platline}  by 5
                \item[] [Time, Velocity, x, y, z]

                \end{itemize} 
    \item \textbf{write\_to\_hdf5}
    \item[] Saves all data in HDF5 format


\end{itemize}

\subsection{main.py}
\label{PythonMain}
Creates the p[roper file structure and ensures all functions save data into the proper location  
The color of links can be changed to your liking using:


\section{FDS2UnityVR}

{\small\verb!\hypersetup{urlcolor=red}!}, or

{\small\verb!\hypersetup{citecolor=green}!}, or

{\small\verb!\hypersetup{allcolor=blue}!}.

\noindent If you want to completely hide the links, you can use:

{\small\verb!\hypersetup{allcolors=.}!}, or even better: 

{\small\verb!\hypersetup{hidelinks}!}.

\noindent If you want to have obvious links in the PDF but not the printed text, use:

{\small\verb!\hypersetup{colorlinks=false}!}.
