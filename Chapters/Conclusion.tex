% Chapter Template

\chapter{Conclusion and Future Work} % Main chapter title

\label{Chapter4} % Change X to a consecutive number; for referencing this chapter elsewhere, use \ref{ChapterX}

Updating the tools available for fire fighter training is a current point of interest for the forestry service. It also has the capability to save life's if done effectively. Our goal with this project was to  design a generalized proof of concept that would be able to be build on to improve training and safety for firefighters. \par
We were able to show with trivial increase to the amount of time for the current pipeline of running and visualizing a simulation, we  were able to substantial lower the total about of memory needed to store pertinent data to be visualized. While also improving the fidelity of the visualization we are able to allow firefighters to experience simulations with out the risks of a real fire. This allows for each user to have a larger mental set of fire environments they have experienced before being put in a dangerous situation, witch will allow them to make more accurate assessments of real life situations and be safer overall. \par

In the future further research in to improving or implementing new thresholding and clustering techniques might have the capability to further optimize our pipeline. As for future capabilities for unity implementation of additional libraries to allow for these projects to run on the Oculis platform would lower the cost of equipment needed for each training facility, additionally further research in to utilizing Unitys WebVR would allow for the system to be hosted online and accessed remotely again lowing the total cost of the equipment needed for each training facility    

