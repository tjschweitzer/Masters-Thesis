% Chapter Template

\chapter{Conclusion and Future Work} % Main chapter title

\label{Chapter4} % Change X to a consecutive number; for referencing this chapter elsewhere, use \ref{ChapterX}

Updating the tools available for fire fighter training is a current point of interest for the forestry service. It also has the capability to save lives if done effectively. Our goal with this project was to design a generalized proof of concept that would be able to be built on to improve training and safety for firefighters. The use of unity’s development platform allows us to compile and gives us the possibility to make available for distribution a self-contained system to help improve training for firefighters. \par
We were able to substantially lower the total amount of memory needed to store pertinent data, with only a trivial increase to the amount of processing time, for the current pipeline of running and visualizing a simulation.While  improving the fidelity of the visualization we are able to allow firefighters to experience simulations without the risks of a real fire. This allows  each user to have a larger mental set of fire environments they have experienced before being put into a dangerous situation, which will allow them to make more accurate assessments in real life situations and be safer overall. \par
The unity stage of our pipeline is currently designed for use with HTC,Steam’s HMD, or no headset. A separate version of the system could be compatible with other brands of HMD with minimal changes to the configuration, further research into this would be needed. \par
In the future, further research into improving or implementing new thresholding and clustering techniques may be able to further optimize our pipeline. As for future capabilities, unity implementation of additional libraries would allow for these projects to run on the Oculus platform, which would lower the cost of equipment needed for each training facility. Additionally further research into utilizing Unity’s WebVR would allow for the system to be hosted online and accessed remotely; again lowering the total cost of the equipment needed for each training facility


