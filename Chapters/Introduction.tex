% Chapter Template

\chapter{Introduction} % Main chapter title

\label{ChapterX} % Change X to a consecutive number; for referencing this chapter elsewhere, use \ref{ChapterX}

%----------------------------------------------------------------------------------------
%	SECTION 1
%----------------------------------------------------------------------------------------

\section{Motivation}

Simulation modeling has become an invaluable tool for emulating smoke and heat release from fires. Its use for training urban and wildland firefighters has the capability to be immeasurably useful as it eliminates the danger of training in hazardous conditions. For example, in 2020 the United States Fire Administration (U.S.F.A.) reported that there were a total of 66 firefighter fatalities (this does not include SARS CoV-2 related deaths). Twelve of those fatalities were at the scene of wildland or outdoor fires \cite{FireFatalities}.

When looking at wildland fire incidents with multiple fatalities related to fire behavior, wind is noted to be a major contributing factor to the fatalities. Some notable examples of this are the Yarnell Hill Fire which had 19 fatalities \cite{yarnellhillfire_2013}, the El Dorado Incident which had one fatality \cite{ElDorado}, the Twisp River Fire which had three fatalities \cite{Twisp}, and the Mann Gulch Fire which had 13 fatalities \cite{mannGulch}. While it is impossible to definitively state that additional training could have prevented these fatalities, it has been shown by B. Wiederhold et. all that stress responses in virtual reality had a high level of correlation to real life stimulus  \cite{Wiederhold200310AI}.

\par
A training environment in virtual reality will allow anyone ranging from brand new  firefighters to fire bosses to experience dynamic fire environments. An advantage with utilizing virtual environments is that we are able to create an unlimited number of unique situations to train firefighters. The more unique environments and situations we are able to expose firefighters to, the better prepared they will be to deal with a fire in an emergency situation. Additionally, virtual reality (VR) is rapidly improving to the point where some VR head mounted displays (HMD) can cost under \$300 and only need to be connected to a phone application to run \cite{Quest}.   \par

Steven G Wheeler, Hendrik Engelbrecht, and Simon Hoermann reviewed six articles pertaining to the use of firefighter training in virtual reality \cite{Wheeler2021}. In each experiment they showed that VR training out performed the control groups and performed at least as well as other traditional training methodologies.  \par

Current visualization software, such as  Smokeview or Paraview, allow the users to change their vantage point and manually filter out what information may be important. A variety of types of data are needed to build an accurate simulation such as physical  obstructions, topography, fuel types, ignition points, wind speed, and wind direction.Topography and fuel type information can all be directly pulled from databases like LANDFIRE. Wind speed and direction data can be obtained from NOAA and visualized directly or put into a simulation model to estimate future wind data [FF17]. While local wind behavior during a fire is complex and dynamic, we are able to model it using the well validated reaction chain and computational fluid dynamic simulator from the National Institute of Standards and Technology (NIST) called Fire Dynamic Simulator (FDS) \cite{FDSValid}.  FDS allows us to generate discrete estimates on a structured grid of wind vectors, temperature, and air density. Improvements in computational hardware and available algorithms has made visualizing these simulations in a digestible manner a tangible goal. 
\par






%-----------------------------------
%	SECTION 2
%-----------------------------------
\section{Graphing Techniques}

Currently there are many unique techniques used for visualizing vector fields in 2-D or 3-D spaces. These can be sub-categorized as textured-based techniques (line integral convolution), glyph based (arrow graphs), or particle tracing (streamlines). All of these approaches have unique benefits but come with a similar set of drawbacks that our methodology looks to resolve.  \par

Line integral convolution (LIC) simulates surface oil patterns in wind tunnel experiments. To do this, a texture of white noise is applied to the domain of an area. This can be a flat plane at any orientation, or more commonly, it is a texture mapped onto a 3-D object. Next a one dimensional convolution with a kernel filter, that is in the direction of the vector field is applied to the white noise, then finally the kernel output is normalized \cite{Cabral1993}. The resulting intensity of LIC pixels is recorded and this provides a visualization of strongly correlated streamlines. A large drawback when it comes to LIC is that line brightness is not indicative of velocity due to the local normalization that occurs. Additionally, the texture will occlude any part of the object behind it; as well as, the small stream lines do not indicate directionality, i.e., a line going left to right looks identical to a line going right to left \cite{LIC}. \par

Hedgehog, or arrow plots, are visualized by inserting glyphs as each cell of data in a vector field. Glyphs can be represented by a variety of 3-D objects, but arrows are primarily used. The orientation of the glyphs can be used to indicate directionality, while the objects scale and color can represent other scalar values. These plots can also be visualized in 2-D or 3-D. The main advantage to this type of graph is the ease of implementation and ability to be understood quickly. The drawback with this technique is when represented in a 3-D area, the visual becomes quite busy if each point in a vector field is represented. Additionally, when the data being represented is time dependent, then the previous time-step glyphs are just replaced with the new glyph. This forces the user to mentally compare the changes and make assumptions based on that. \par

Particle tracing tends to be a larger subcategory as the techniques used change drastically based on several factors. If the vector field has steady state flow, meaning it does not change in time, we refer to these as streamlines. Pathline is the term used if the vector field is time dependent. We also have streaklines, where a line is created from all particles that pass through a given point. Placement of these points can be predetermined, random, or placed to form a designated shape like a ribbon or a cylinder. To calculate the flow of a particle through a volume, the ordinary differential equation solvers (ODE’s) are used to move the particles through velocity fields. ODE solvers with dynamic time stepping, like Runge-Kutta 4(5), ensures that we have a more accurate model then one with a discrete time step like Euler’s \cite{Teitzel1997}. These visualizations can be textured to indicate velocity at a set point along the line. They have the same demerits, as previously discussed with LIC; there is no clear start or end to each line and resolving areas of interest is difficult without previous knowledge of the vector field. \par

All of these visualization techniques generally have a large drawback, all information in the field is visualized at once, forcing the user to mentally compare the change between time steps, all while having other data being occluded by data closer to the user’s vantage point. Solutions often involve the user having to  preselect where they assume that the most important information will be, which is difficult to do without prior visualization. 

Our goal is to develop an optimized way to visualize pathlines in vector fields focused on improving firefighter training. For fire simulations the areas we will be interested in are areas with turbulent air flow that can cause unpredictable fire behavior.  Using the reynolds number allows us to have a unitless value to indicate air turbulence, with this we can calculate pathlines that cross these areas of interest.










