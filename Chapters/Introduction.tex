% Chapter Template

\chapter{Introduction} % Main chapter title

\label{ChapterX} % Change X to a consecutive number; for referencing this chapter elsewhere, use \ref{ChapterX}

%----------------------------------------------------------------------------------------
%	SECTION 1
%----------------------------------------------------------------------------------------

\section{Motivation}

Simulation modeling has become an invaluable tool for emulating smoke and heat release from fires. Its use for training urban and wildland firefighters has the capability to be immeasurably useful as it mediates the danger of training inside of a classroom environment. In 2020 the United States Fire Administration (U.S.F.A.) reported that there were a total of 102 firefighter fatalities; 12 fatalities were at the scene of wildland or outdoor fires. \cite{FireFatalities}

Steven G Wheeler, Hendrik Engelbrecht, and Simon Hoermann, reviewed six articles pertaining to the use of firefighter training in virtual reality (VR).\cite{Wheeler2021}  In each experiment it showed that VR training out performed the control groups and performed at least as well as other traditional training methodologies. Another advantage with utilizing virtual environments is that we are able to create an unlimited number of unique situations for the firefighters to train on. The more unique environments and situations we are able to expose firefighters to, the better prepared they will be to deal with a fire in an emergency situation. 
\par
Current visualization processes, like Smokeview or Paraview, allow the users to change their vantage point and manually filter out what information may be important. Determining points of interest in a wind field is based on several factors including obstructions, topography, fuel types, ignition points and wind speed and direction. Topography and fuel type information can all be directly pulled from databases like LANDFIRE, while wind speed and direction data can be obtained from NOAA and used directly or  put into a Neural Network to generate future wind data (\cite{WindNN}). While local wind behavior during a fire is dynamic, we are able to model it using the highly validated program from National Institute of Standards and Technology (NIST) called Fire Dynamic Simulator (FDS). \color{red}[FDS Reference and stat?]\color{black} This allows us to have discrete data in terms of wind vectors, temperature, and air density. Visualizing this data in a digestible manner has become a tangible goal with the content improvements with computational hardware and available algorithms.   




%-----------------------------------
%	SECTION 2
%-----------------------------------
\section{Graphing Techniques}

Currently there are many unique techniques used for visualizing vector fields in 2-D or 3-D space. These can be sub categorized as textured-based techniques (Line Integral Convolution), glyph based (Arrow Graphs), or particle tracing (Streamlines). All of these approaches have unique benefits but come with a similar set of drawbacks that our methodology looks to resolve. \par

Line Integral Convolution (LIC) simulates surface oil patterns in wind tunnel experiments. To do this, a texture of white noise is applied to the domain of an area. This can be a flat plane at any orientation, or more commonly, it is a texture mapped onto a 3-D object. Then a 1-Dimensional convolution with a kernel filter, that is in the direction of the vector field is applied to the white noise then finally the kernel output is normalized.\cite{Cabral1993} The resulting intensity of LIC pixels is recorded and this provides a visualization of strongly correlated streamlines. A large drawback when it comes to LIC is that line brightness is not indicative of velocity due to the local normalization that occurs.  Additionally, the texture will occlude any part of the object behind it; as well as, the small stream lines do not indicate directionality.  line going left to right looks identical to a line going right to left. \cite{LIC}\par

Hedgehog, or arrow plots, are visualized by inserting glyphs as each cell of data in a vector field. Glyphs can be represented by a variety of 3-D objects, but arrows are primarily used. The orientation of the glyphs can be used to indicate directionality, while the objects scale and color can represent other scalier values if desired. These plots can also be visualized in 2-D or 3-D. The main advantage to this type of graph is the ease of implementation and  ability to be understood quickly. The drawback with this technique is when represented in a 3-D area, the visual becomes quite busy if each point in a vector field is represented. Additionally, when the data being represented is time dependent, then the previous time-step glyphs are just replaced with the new glyph. This forces the user to mentally compare the changes and make assumptions based on that. (Citation Needed) 
\par
Particle tracing tends to be a larger subcategory as the techniques used change drastically based on several factors. If the vector field has steady state flow, meaning it does not change as time progresses, we refer to these as streamlines, or path lines, if the vector field is time dependent. We also have streak lines, where a line is created from all particles that pass-through a given point. Placement of these points can be predetermined, random, or placed to form a designated shape like a ribbon or a cylinder. To calculate the flow of a particle through an area, the use of ordinary differential equations (ODE’s) is used. ODE with dynamic time stepping, like Runge-Kutta 4(5), ensures that we have proper sensitivity for a more accurate model then one with a discrete time step like Euler’s. These visualizations can be textured to indicate velocity at a set point along the line. They also have the same deterrence, as previously discussed with LIC, where there is no clear start or end to each line and determining the areas of interest is difficult without previous knowledge of the vector field. (Citation Needed) 
\par
All of these visualization techniques  generally have a large drawback where all information in the field is visualized at once forcing the user to  mentally compare the change  between time steps, all while having other data being occluded by data closer to the user’s vantage point. Or requiring the user to have pre-selected where it is assumed that the most important information will be. (Citation Needed) 


Our goal is to optimize the way these are visualized as well as calculate points of interest in the simulation and calculate appropriate starting points for the streak lines to intersect these points of interest. For firefighting training points of interest are areas where there is turbulent air flow that can cause unpredictable fire behavior while inserting pathlines in areas of laminar flow as well.


%-----------------------------------
%	SECTION 3
%-----------------------------------
\section{Wind Turbulence}

This is where information about RE will go maybe if needed





