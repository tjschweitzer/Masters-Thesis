% Chapter Template

\chapter{Introduction} % Main chapter title

\label{ChapterX} % Change X to a consecutive number; for referencing this chapter elsewhere, use \ref{ChapterX}

%----------------------------------------------------------------------------------------
%	SECTION 1
%----------------------------------------------------------------------------------------

\section{Motivation}

Simulation modeling has become an invaluable tool for emulating smoke and heat release from fires. It’s use for training urban and wildland fire fighters has the capability to be immeasurably useful as it mediates the danger of training inside of a classroom environment. In 2020 the United States Fire Administration (U.S.F.A.) reported that there was a total of 102 firefighter fatalities with 12 fatalities were at the scene of wildland or outdoor fires. \cite{FireFatalities}

\cite{Wheeler2021} reviewed six articles pertaining the use of firefighter training in virtual reality (VR) in each experiment it showed that training in VR out preformed the control groups and performed at least as well as other traditional training methodologies. Another advantage with utilizing virtual environments is that we are able to create an unlimited number of unique situations for the firefighters to train on. The more unique environments and situations were able to expose firefighters, the better prepared to deal with a fire in an emergency situation. Current visualization processes like Smokeview or Paraview, allow the users to change their vantage point and manually filter out what information that may be important. Determining points of interest in a wind field is based on several factors including obstructions, topography, fuel types, ignition points and wind speed and direction. 
Topography and fuel type information can all be directly pulled from databases like LANDFIRE, while wind speed and direction data can be obtained from NOAA and used directly or even put in to a Neural Network to generate future wind data (\cite{WindNN})  While local wind behavior during a fire is dynamic, we are able to model it using the highly validated  program from National Institute of Standards and Technology (NIST) called Fire Dynamic Simulator (FDS). [FDS Reference and stat?] This allows us to have discrete data in terms of wind vectors, temperature, air density. Visualizing this data in a digestible manner has become a tangible goal with the content improvements with computational hardware and available algorithms.




%-----------------------------------
%	SECTION 2
%-----------------------------------
\section{Graphing Techniques}

Currently there are many unique techniques used for visualizing vector fields in 2-D or 3-D space. These can be sub categorized as a textured-based techniques (Line Integral Convolution), glyph based (Arrow Graphs), or particle tracing (Streamlines). All of these approaches have unique benefits but come with a similar set of drawbacks that our methodology looks to resolve.\par

Line Integral Convolution (LIC) simulates surface oil patterns in wind tunnel experiments. To do this a texture of white noise is applied to the domain of an area, this can be a flat plane at any orientation or more commonly it is a texture mapped to a 3-D object.  Then a 1-dimensional Convolution with a kernel filter, that is in the direction of the vector field and finally the kernel output is normalized.\cite{Cabral1993} The resulting intensity of LIC pixels are recorded this provides visualization of strongly correlated streamlines. A large drawback when it comes to LIC is that line brightness is not indicative of velocity due to the local normalization that occurs additionally the texture will occlude any part of the object behind it as well as the small streamlines do not indicate directionality, a line going left to right looks identical to a line going right to left. \cite{LIC}\par

Hedgehog or arrow plots are visualized by inserting glyphs as each cell of data in vector field. Glyphs can be represented by a verity of 3-D objects, but arrows are primarily used. The orientation of the glyphs can be used to indicate directionality while the objects scale and color can represent other scalier values if desired, these plots can also be visualized in 2-D or 3-D. The main advantage to this type of graph is their ease of implementation and their ability to be understood quickly. The drawback with this technique is that visual is quite busy if each point in a vector field is represented and with each glyph representing an independent value at a specific time step, individual points do not represent the trends or previous values.  {Citation Needed}
\par
Particle tracing tends to be a larger subcategory as the techniques used change drastically based on several factors. If the vector field has steady state flow (does not change as time progresses) we refer to these as streamlines or a path line if the vector field is time dependent. We also have streak lines where a line is created from all particles that pass-through a given point. Placement of these points can be predetermined, random, or placed to form a designated shape like ribbon or a cylinder. To calculate the flow of a particle though an area the use of ordinary differential equations (ODE’s) is used, and ODE with dynamic time stepping like Runge-Kuta 4(5) ensures that we have proper sensitivity for a more accurate model then one with a discrete time step like Euler’s. These visualizations can be textured to indicate velocity at a set point along the line, they also have the same determent as previously discussed with LIC where there is no clear start or end to each line and determining the areas of interest is difficult without previous knowledge of the vector field.   {Citation Needed}
\par
With all of these visualization techniques have a tendency to have a large drawback where all information in the field is visualized at once forcing the user to have to mentally compare the change in between time steps, while having other data being occluded by data closer to the user’s vantage point. Or requiring the user to have pre-selected where it is assumed that more important information will be.  {Citation Needed}


Our goal is to optimize the way these are visualized as well as calculate points of interest in the simulation and calculate appropriate starting points for the streak lines to intersect these points of interest. For firefighting training points of interest are areas where there is turbulent air flow that can cause unpredictable fire behavior while inserting pathlines in areas of laminar flow as well.


%-----------------------------------
%	SECTION 3
%-----------------------------------
\section{Wind Turbulance}

This is where information about RE will go

%-----------------------------------
%	SECTION ?
%-----------------------------------
\section{Homeless Text}
can be used to display a large verity of graphs graphs created by {} where the directionality of the vector is indicated by change in the thickness of the line segments and length indicated velocity. This type of graph is still generally used today just without the change in line thickness to indicate  directionality. \cite{567777} The benefits of this are quick views of an area highlighting areas of high and low velocity, while not showing clear start and end points of the streams, additionally for an optimum plot we tend to need to have the orientation and location of the plane beforehand. 



